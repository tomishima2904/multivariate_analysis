\section{$\S1.13$説明変数の選択}
これまでの議論では,回帰モデルに含まれる説明変数$x_{1}, \cdots, x_{p}$は定められたものとして回帰係数や予測値を計算してきた. しかし, 実際に現象を分析する場合には, 目的変数$y$に影響を及ぼすかもしれないと考えられる変数は多数あり, その中から次のようなことを考慮し, しかも実質科学的にも重要な変数を取り上げるのが普通である. 

\begin{enumerate}
  \item 回帰モデルに無駄な変数(真の回帰係数が0であるような変数)が含まれる場合, 回帰係数の推定値$\hat{a}_j$, 目的変数の予測値$Y_0$は不偏であるが, 誤差分散の推定値$\operatorname{V}_e$の自由度$n-p-1$が小さくなり, $\hat{a}_i$や$Y_0$の推定精度が悪くなる. 
  \item 必要な変数(真の回帰係数が0でない変数)が回帰モデルの中から漏れている場合, 回帰係数の推定値, 目的変数の予測値は偏りをもち, また誤差分散の推定値$V_e$は過大評価になる. 
  \item 説明変数の中に互いに相関が高い変数が含まれる場合には, 分散共分散行列$V=(s_{jl})$の行列式が0に近くなるため, 逆行列の要素$s^{jj}=V_{jj}/|V|$が大きくなり, 回帰係数の推定精度は悪くなる. 特に説明変数の中のひとつと残りの変数との重相関係数が$\operatorname{R}=1$のときには分散共分散行列$(s_{jl})$の行列式は0になり逆行列が存在しないため, 回帰係数の推定値$\hat{a}_j$は得られない. このような場合{\bm 多重共線性}の問題があると言う. 
\end{enumerate}

説明変数の候補の中から最良な変数を選択して回帰式を求めるための統計的方法として次のような方法が提案されている. 

\begin{description}
  \item[(1) 総あたり法]\mbox{}\\
  $p$個の説明変数の候補の中から$1\sim p$個の変数の可能なすべての部分集合に対応する$2^p-1$通りの回帰モデルを検討する方法. $p$が大きくなると場合の数は急速に大きくなり, 計算時間が膨大になる. 
  \item[(2) 前進選択法]\mbox{}\\
  説明変数が1つも含まれない場合からスタートして, 次のような手順で変数を1つずつ増加させる. 
  \begin{enumerate}
    \item 目的変数$y$との単相関が最大(すなわち, 1つずつ順番に変数を採用してみて回帰式を計算したとき, 回帰係数の検定のためのtの絶対値またはF値が最大)の変数を選び, 回帰係数がゼロであるという仮説の検定をして仮説が棄却されなければどの変数も回帰モデルに含めない. 仮説が棄却されればこの変数を取り込んで次のステップに進む. 
    \item 既に入っている変数に加えて残りの変数を1つずつ順番に採用してみて偏相関係数が最大(回帰係数検定のためのtの絶対値またはF値が最大)の変数を選ぶ. 選ばれた変数に対する回帰係数が0であると言う仮説の検定をおこない, 仮説が棄却されなければ終了. 仮説が棄却されれば選ばれた変数を取り込んで次のステップへ進む. 
    \item 回帰式を計算する. もしモデルに全ての変数が含まれていれば終了. そうでなければステップ2に戻る
  \end{enumerate}
  \item [(3) 後退消去法]\mbox{}\\
  説明変数の候補すべてが含まれた状態からスタートして次のような手順で変数を1つずつ減少させる. 
  \begin{enumerate}
    \item モデルに含まれている変数の各々に対する回帰係数検定のためのtまたはF値を計算し, その中の絶対値が最小となる変数を選ぶ. 回帰係数が0であるという仮説が棄却されなければその変数を落として次のステップへ進む. 棄却されれば終了. 
    \item もしモデルに含まれる変数がなくなっていれば終了. そうでなければ回帰式を計算し直してステップ1に戻る. 
  \end{enumerate}
  \item [(4) 逐次法]\mbox{}\\
  前進選択法では1度入った変数は落とされることがないという点を改良して, 次のような手順で変数を増減させる. 
  \begin{enumerate}
    \item 目的変数$y$との単相関が最大の変数を選ぶ. 選ばれた変数に対する回帰係数が0であるという仮説の検定をおこない, 棄却されなければどの変数も回帰モデルに含めない. 棄却されればこの変数を取り込んで次のステップに進む. 
    \item 既に入っている変数に加えて残りの変数を1つずつ順番に採用してみて偏相関係数が最大の変数を選ぶ. 回帰係数が0であるという仮説が棄却されなければ終了. 棄却されれば選ばれた変数を取り込んで次のステップに進む. 
    \item 回帰式を計算して各変数について回帰係数の検定をおこない, F値が最小になる変数について仮説が棄却されなければその変数をおとす.
    \item すべての変数が取り込まれていれば終了. そうでなければステップ2に戻る. 
  \end{enumerate}
\end{description}

上記4つの方法において, 回帰係数の検定は次のように行う. 
$p$個の変数を含むモデルでの変数$x_j$に対する回帰係数が0という仮説$\operatorname{H}_0: a_j=0$の検定は, 前節$\clubsuit$式(19)より, 
\begin{align*}
  |t|= \frac{|\hat{a}_j-a_j^{(0)}|}{\sqrt{\frac{s^{jj}\operatorname{V}_e}{n}}} \geq t_{\alpha}(n-p-1)
\end{align*}
で, $a_j^{(0)}=0$とおき, 
\begin{align}
  t = \frac{\hat{a}_j}{\sqrt{s^{jj}\operatorname{V}_e/n}}
\end{align}
の値を求めて, 自由度$n-p-1$のt分布の限界値と比較し, $|t|\geq t_\alpha(n-p-1)$ならば仮説を棄却, $|t|\leq t_\alpha(n-p-1)$ならば仮説を採択する. また, ある$x$が自由度$n$のt分布に従うとき, $x^2$は自由度$(1, n)$のF分布に従うことから, 
\begin{align}
  \label{eq:f_dist}
  \operatorname{F} = \frac{\hat{a}^2_{j}}{s^{jj}\operatorname{V}_e/n}
\end{align}
の値を求めて, 自由度$(1, n-p-1)$のF分布の限界値と比較し, $|\operatorname{F}|\geq \operatorname{F}_{n-p-1}^1(\alpha)$ならば仮説を棄却することで同じ結果を得ることができる. 

\section*{付録 多重共線性}
% Multicollinearity
説明変数が2つの回帰モデル$y=\hat{a}_0+\hat{a}_1x_1+\hat{a}_2x_2$を例に, 多重共線性の問題について考える. 回帰係数$\hat{a}_1$を求める式は, $\S1.5$より, 
\begin{eqnarray}
  \label{eq:hatA1_regre}
  \hat{a}_1
  &=&\frac{
    \left|
      \begin{array}{cc}
        s_{y1} &s_{12} \\
        s_{y1} &s_{22}
      \end{array}
    \right|
  }
  {
    \left|
      \begin{array}{cc}
        s_{11} &s_{12} \\
        s_{21} &s_{22}
      \end{array}
    \right|
  }
\end{eqnarray}
である. この式の分母, すなわち, 分散共分散行列$\bm{V}=(s_{jl})$の行列式は, 
\begin{align}
  \notag
  |\bm{V}| 
  &= s_{11}s_{22} - s_{12}s_{21} \\
  \notag
  &= s_{11}s_{22}\left(
    1 - \frac{s_{12}}{\sqrt{s_{11}s_{22}}} \frac{s_{21}}{\sqrt{s_{11}s_{22}}}
  \right)\\
  \label{eq:determinant_V}
  &= s_{11}s_{22}(1-r^2)
\end{align}
となる. このとき$r$は, $x_1$と$x_2$の相関係数である. 

$x_{1}$と$x_{2}$の相関が高い, すなわち相関係数$r$が$\pm 1$に近い値をとるとき, 分散共分散行列の行列式は0に近くなり, 回帰係数$\hat{a}_1$は式\eqref{eq:hatA1_regre}の分子の行列式($s_{y1}s_{22}-s_{12}s_{y1}$)の変化に大きく影響されることがわかる. したがって, 変数の値の変化によって回帰係数の推定値が大きく変わるため, 係数推定値の分散が大きくなり, 推定結果の信頼性が落ちる. また, 特に相関係数が$r=\pm 1$(線形従属)のとき, 分散共分散行列の行列式は0となり, それを分母にもつ式$\eqref{eq:hatA1_regre}$は発散してしまし, 求めることができない. 

重回帰分析では,候補となる説明変数の間に相関がないことを確認し,相関が見られた場合にはその 説明変数を外すことで,多重共線性の問題を回避する必要がある.
