\documentclass[a4paper,11pt,dvipdfmx]{jsarticle}

\usepackage[dvipdfmx]{graphicx}
\usepackage{subcaption}
\usepackage{url}
\usepackage{bm}
\usepackage{amsmath,amssymb}
\usepackage{ascmac}
\usepackage{listings,jlisting}

\lstset{%
  language={C},
  basicstyle={\small},
  identifierstyle={\small},
  commentstyle={\small\itshape},
  keywordstyle={\small\bfseries},
  ndkeywordstyle={\small},
  stringstyle={\small\ttfamily},
  frame={tb},
  breaklines=true,
  columns=[l]{fullflexible},
  numbers=left,
  xrightmargin=0zw,
  xleftmargin=3zw,
  numberstyle={\scriptsize},
  stepnumber=1,
  numbersep=1zw,
  lineskip=-0.5ex
}

%\captionsetup[subfigure]{labelformat=simple}
\renewcommand{\thesubfigure}{(\alph{subfigure})}

\newcommand{\qed}{\hfill$\Box$}
\newcommand{\Proof}{\noindent{\bf 証明}\quad}
\newcommand{\Thm}{\noindent{\bf 定理}\quad}
\newcommand{\Exercise}{\noindent{\bf 例題}\quad}
\newcommand{\Ans}{\noindent{\bf 解答}\quad}
\makeatletter
\newcommand{\figcaption}[1]{\def\@captype{figure}\caption{#1}}
\newcommand{\tblcaption}[1]{\def\@captype{table}\caption{#1}}
\makeatother

\newcommand{\bhline}[1]{\noalign{\hrule height #1}}  
\newcommand{\bvline}[1]{\vrule width #1}  

\begin{document}

\title{\vspace{-2cm}補足資料}
\author{富島諒}
\date{\today}

\maketitle

回帰の変動は
\begin{align*}
  \sum_{i=1}^{n}(Y_i-\bar{Y})^2
  &= \sum_{i=1}^{n}\left\{
    (\hat{a}_0 + \hat{a}_{1}x_{1i} + \cdots + \hat{a}_{p}x_{pi})
    - (\hat{a}_0 + \hat{a}_{1}\bar{x}_1 + \cdots + \hat{a}_{p}\bar{x}_p)
  \right\}^2 \\
  &= \sum_{i=1}^{n} \left\{
    \hat{a}_1(x_{1i}-\bar{x}_1) + \cdots + \hat{a}_p(x_{pi}-\bar{x}_p)
  \right\}^2 \\
  &= \sum_{i=1}^{n} \sum_{j=1}^p \hat{a}_j^2(x_{ji} - \bar{x}_{j})^2
  + \sum_{i=1}^n\sum_{j=1}^p\sum_{l=1}^p \hat{a}_j\hat{a}_l(x_{ji}-\bar{x}_{j})(x_{li}-\bar{x}_l) \qquad (j\neq l)\\
  &= n\sum_{j=1}^p \hat{a}_j^2s_{jj} 
  + n\sum_{j=1}^p\sum_{l=1}^p\hat{a}_j\hat{a}_ls_{jl} 
\end{align*}
$\operatorname{V}(\hat{a}_j)=\frac{\sigma^2}{ns_{jj}}$, かつ, 説明変数$x_j, x_l$は互いに独立であることを用いると, 
\begin{align*}
  \sum_{i=1}^{n}(Y_i-\bar{Y})^2
  &= \sum_{j=1}^p\frac{\hat{a}_j^2\sigma^2}{\operatorname{V}(\hat{a}_j)} \\
  \frac{1}{\sigma^2}\sum_{i=1}^{n}(Y_i-\bar{Y})^2 
  &= \sum_{j=1}^p\frac{\hat{a}_j^2}{\operatorname{V}(\hat{a}_j)} \\
  &= \sum_{j=1}^p\left( \frac{\hat{a}_j}{\sqrt{\operatorname{V}(\hat{a}_j)}} \right)^2
\end{align*}
$u = \frac{\hat{a}_j -a_j}{\sqrt{\operatorname{V}(a_j)}}\sim \operatorname{N}(0, 1)$において, $a_j=0, j=1, \cdots, p$と仮定すると, 
\begin{align*}
  u= \frac{\hat{a}_j}{\sqrt{\operatorname{V}(a_j)}} \sim \operatorname{N}(0, 1)
\end{align*}
なので, 
\begin{align*}
  \sum_{j=1}^p\left( \frac{\hat{a}_j}{\sqrt{\operatorname{V}(\hat{a}_j)}} \right)^2
\end{align*}
は自由度$p$の$\chi^2$分布に従う. すなわち,
\begin{align*}
  \sum_{i=1}^n\frac{(Y_i - \bar{Y})^2}{\sigma^2}
\end{align*}
は自由度$p$の$\chi^2$分布に従う.

\end{document}