\documentclass[dvipdfmx]{jsarticle}
\usepackage{colortbl}
\usepackage{amsmath}
\usepackage[dvipdfmx]{graphicx}
\usepackage{ascmac}
\begin{document}
\title{重回帰分析}
\author{B4 鈴木 克樹}
\maketitle

\section{線形重回帰}
目的変数$y$に対し,説明変数として$y$に関係の深そうな変数が複数(2つ以上)ある場合がある.このとき,目的変数と複数の説明変数の関係を表すモデルを考える.これを重回帰分析という.
重回帰分析のモデルには様々なものを推測できるが,ここではもっとも簡単な線形モデルを考える.

表1のように,1つの目的変数$y$と$p$個の説明変数$x_1, x_2, ... , x_p$について$n$個のデータが与えられたとする.
\begin{table}[htb]
  \begin{center}
    \caption{重回帰分析におけるデータ}
    \begin{tabular}{|>{\columncolor[rgb]{1.0, 0.8, 0.5}} c|c|c|}
      \hline
      \rowcolor[rgb]{1.0, 0.8, 0.5}
      データ番号 & 説明変数 & 目的変数                                                                                                                \\
      \hline
      \rowcolor[rgb]{1.0, 0.8, 0.5}
                 & $y$      & $x_1 \; \;  x_2 \; \; \cdots \cdots \cdots \cdots \cdots \cdots \cdots \cdots \; \; x_p$                                \\
      \hline
      $1$        & $y_1$    & $x_{11} \; x_{21} \; \cdots \cdots \cdots \cdots \cdots \cdots \cdots \cdots \; x_{p1}$                                 \\
      $2$        & $y_2$    & $x_{12} \; x_{22} \; \cdots \cdots \cdots \cdots \cdots \cdots \cdots \cdots \; x_{p2}$                                 \\
      $\vdots$   & $\vdots$ & $\vdots \; \; \; \; \; \vdots \; \; \; \; \; \; \; \; \; \; \; \; \; \; \; \; \; \; \; \; \; \; \; \; \; \; \; \; \; \;
      \; \; \; \; \; \; \; \; \; \; \; \; \; \; \; \vdots$                                                                                            \\
      $i$        & $y_i$    & $x_{1i} \; x_{2i} \; \cdots \cdots \cdots \cdots \cdots \cdots \cdots \cdots \; x_{pi}$                                 \\
      $\vdots$   & $\vdots$ & $\vdots \; \; \; \; \; \vdots \; \; \; \; \; \; \; \; \; \; \; \; \; \; \; \; \; \; \; \; \; \; \; \; \; \; \; \; \; \;
      \; \; \; \; \; \; \; \; \; \; \; \; \; \; \; \vdots$                                                                                            \\
      $n$        & $y_n$    & $x_{1n} \; x_{2n} \; \cdots \cdots \cdots \cdots \cdots \cdots \cdots \cdots \; x_{pn}$                                 \\
      \hline
    \end{tabular}
  \end{center}
\end{table}

$x_1, x_2, \cdots, x_p$から$y$の値を予測するとき,$x_1, x_2, \cdots, x_p$と$y$の関係を示す1つの数式モデルを設定しなければならない.この数式モデルを(1)式のように与える.$\epsilon_i$は$x_{1i}, x_{2i}, \cdots, x_{pi}$だけでは説明しきれない部分の予測誤差である.
% (1)
\begin{equation}
  y_i = a_0 + a_1x_{1i}+a_2x_{2i}+\cdots + a_p x_{pi}+\epsilon_i
\end{equation}
この式を\textbf{線形重回帰モデル(linear multiple recogression model)}と呼ぶ.単回帰分析の場合,$(x,y)$平面上のn個の点の集まりに直線をあてはめるが,重回帰分析の場合は$(x_1, x_2, \cdots, x_p, y)$の$p+1$次元空間でのn個の点の集まりに対し,p次元超平面をあてはめる.そうすることで,説明変数の値を$x_1, x_2, \cdots, x_p$から目的変数の値$y_i$を予測する.p次元超平面を(2)式に示す.
% (2)
\begin{equation}
  y =  a_0 + a_1x_{1}+a_2x_{2}+\cdots + a_p x_{p}
\end{equation}
このときの予測誤差$\epsilon_i\;(i=0, 1,\cdots,n)$は(1)式を変形して(3)式のように表される.
% (3)
\begin{equation}
  \epsilon_i = y_i - (a_0 + a_1x_{1i}+a_2x_{2i}+\cdots + a_p x_{pi})
\end{equation}
この予測誤差を小さくするために,各予測誤差の平方和を最小にするような$a_0, a_1, \cdots, a_p$を
求めることにし,それを$\hat{a_0},\hat{a_1},\cdots,\hat{a_p}$と書く.また,各予測誤差の平方和を(4)式に示す.
% (4)
\begin{equation}
  F(a_0,a_1,\cdots,a_p)=\sum_{i=1}^{n}\epsilon_i^2=\sum_{i=1}^{n}\{ y_i - (a_0 + a_1x_{1i}+a_2x_{2i}+\cdots + a_p x_{pi})\}^2
\end{equation}
各予測誤差の平方和を最小にするということは,(3)式を$a_0,a_1,\cdots,a_p$で偏微分したものが0となるような
$\hat{a_0},\hat{a_1},\cdots,\hat{a_p}$を求めることと同値である.つまり(5)式を解けばいい.この解法を\textbf{最小二乗法(least squares method)}という.
% (5)
\begin{equation}
  \begin{cases}
    \displaystyle
    \frac{\partial F(a_0,a_1,\cdots,a_p)}{\partial a_0} = 0 & \\
    \displaystyle
    \frac{\partial F(a_0,a_1,\cdots,a_p)}{\partial a_1} = 0 & \\
    \displaystyle
    \; \; \; \; \; \; \; \; \; \; \; \; \; \; \vdots          \\
    \displaystyle
    \frac{\partial F(a_0,a_1,\cdots,a_p)}{\partial a_p} = 0 & \\
  \end{cases}
\end{equation}

(5)式の左辺を偏微分後の式に変形すると(6)式のようになる.
% (6)
\begin{equation}
  \begin{cases}
    \displaystyle
    \frac{\partial F(a_0,a_1,\cdots,a_p)}{\partial a_0} = -2\sum_{i=1}^n \{ y_i - (a_0 + a_1x_{1i}+a_2x_{2i}+\cdots + a_p x_{pi})\} = 0       & \\
    \displaystyle
    \frac{\partial F(a_0,a_1,\cdots,a_p)}{\partial a_1} = -2\sum_{i=1}^n \{ y_i - (a_0 + a_1x_{1i}+a_2x_{2i}+\cdots + a_p x_{pi})\}x_{1i} = 0 & \\
    \displaystyle
    \; \; \; \; \; \; \; \; \; \; \; \; \; \; \vdots                                                                                            \\
    \displaystyle
    \frac{\partial F(a_0,a_1,\cdots,a_p)}{\partial a_p} = -2\sum_{i=1}^n \{ y_i - (a_0 + a_1x_{1i}+a_2x_{2i}+\cdots + a_p x_{pi})\}x_{pi} = 0 &
  \end{cases}
\end{equation}

(6)式の各方程式の$\displaystyle \sum_{i=0}^n y_i$が含まれている項を右辺に移項し,整理すると(7)式を得ることができる.
% (7)
\begin{equation}
  \begin{cases}
    \displaystyle
    na_0+\left(\sum_{i=1}^n x_{1i}\right)a_1+\cdots+\left(\sum_{i=1}^n x_{pi}\right)a_p=\sum_{i=1}^ny_i                                                & \\
    \displaystyle
    \left(\sum_{i=1}^n x_{1i}\right)a_0+\left(\sum_{i=1}^n x_{1i}^2\right)a_1+\cdots+\left(\sum_{i=1}^n x_{pi}x_{1i}\right)a_p=\sum_{i=1}^ny_i x_{1i}  & \\
    \displaystyle
    \; \; \; \vdots                                                                                                                                      \\
    \displaystyle
    \left(\sum_{i=1}^n x_{pi}\right)a_0+\left(\sum_{i=1}^n x_{1i}x_{pi}\right)a_1+\cdots+\left(\sum_{i=1}^n x_{pi}^2\right)a_p=\sum_{i=1}^ny_i x_{pi} &
  \end{cases}
\end{equation}

(7)式は未知の定数$a_0,a_1,\cdots,a_p$の連立方程式であって,\textbf{正規方程式(normal equation)}と呼ばれている.この式は行列で表現することにより,クラーメルの公式を用いて解くことが可能である.行列形式で表した(7)式を(8)式に示し,クラーメルの公式の説明を(9)式に示す.


% (8)
\begin{equation}
  \left(
  \begin{array}{cccc}
    \displaystyle
    n                         &
    \displaystyle
    \sum_{i=1}^n x_{1i}       &
    \displaystyle
    \cdots                    &
    \displaystyle
    \sum_{i=1}^n x_{pi}         \\
    \displaystyle
    \sum_{i=1}^n x_{1i}       &
    \displaystyle
    \sum_{i=1}^n x_{1i}^2     &
    \displaystyle
    \cdots                    &
    \displaystyle
    \sum_{i=1}^n x_{pi}x_{1i}   \\
    \vdots                    &
    \vdots                    &
    \ddots                    &
    \vdots                      \\
    \displaystyle
    \sum_{i=1}^n x_{pi}       &
    \displaystyle
    \sum_{i=1}^n x_{1i}x_{pi} &
    \displaystyle
    \cdots                    &
    \displaystyle
    \sum_{i=1}^n x_{pi}^2
  \end{array}
  \right)
  \left(
  \begin{array}{c}
    a_0    \\
    a_1    \\
    \vdots \\
    a_p
  \end{array}
  \right)
  =
  \left(
  \begin{array}{c}
    \displaystyle
    \sum_{i=1}^n y_i       \\
    \displaystyle
    \sum_{i=1}^n y_ix_{1i} \\
    \vdots                 \\
    \displaystyle
    \sum_{i=1}^n y_ix_{pi}
  \end{array}
  \right)
\end{equation}

% (9)
\begin{eqnarray}
連立方程式A\boldsymbol{x}&=&\boldsymbol{b}の解は,\nonumber \\
x_i&=&\frac{detA_i}{detA} \nonumber \\
となる.ただし,&x_i&は\boldsymbol{x}の第i成分であり,A_iはAの第i列の部分を\boldsymbol{b}にしたものである.
\end{eqnarray}

クラーメルの公式より,(8)式を解くと(10)式のような結果を得る.
% (10)
\begin{equation}
  \begin{array}{c}
    \hat{a_0} =
    \frac{
      \begin{vmatrix}
        \displaystyle
        \sum_{i=1}^n y_i          &
        \displaystyle
        \sum_{i=1}^n x_{1i}       &
        \displaystyle
        \cdots                    &
        \displaystyle
        \sum_{i=1}^n x_{pi}         \\
        \displaystyle
        \sum_{i=1}^n y_ix_{1i}    &
        \displaystyle
        \sum_{i=1}^n x_{1i}^2     &
        \displaystyle
        \cdots                    &
        \displaystyle
        \sum_{i=1}^n x_{pi}x_{1i}   \\
        \vdots                    &
        \vdots                    &
        \ddots                    &
        \vdots                      \\
        \displaystyle
        \sum_{i=1}^n y_ix_{pi}    &
        \displaystyle
        \sum_{i=1}^n x_{1i}x_{pi} &
        \displaystyle
        \cdots                    &
        \displaystyle
        \sum_{i=1}^n x_{pi}^2
      \end{vmatrix}
    }{
      \begin{vmatrix}
        \displaystyle
        n                         &
        \displaystyle
        \sum_{i=1}^n x_{1i}       &
        \displaystyle
        \cdots                    &
        \displaystyle
        \sum_{i=1}^n x_{pi}         \\
        \displaystyle
        \sum_{i=1}^n x_{1i}       &
        \displaystyle
        \sum_{i=1}^n x_{1i}^2     &
        \displaystyle
        \cdots                    &
        \displaystyle
        \sum_{i=1}^n x_{pi}x_{1i}   \\
        \vdots                    &
        \vdots                    &
        \ddots                    &
        \vdots                      \\
        \displaystyle
        \sum_{i=1}^n x_{pi}       &
        \displaystyle
        \sum_{i=1}^n x_{1i}x_{pi} &
        \displaystyle
        \cdots                    &
        \displaystyle
        \sum_{i=1}^n x_{pi}^2
      \end{vmatrix}
    }

    \\
    \\

    \hat{a_1} =
    \frac{
      \begin{vmatrix}
        \displaystyle
        n                      &
        \displaystyle
        \sum_{i=1}^n y_i       &
        \displaystyle
        \cdots                 &
        \displaystyle
        \sum_{i=1}^n x_{pi}       \\
        \displaystyle
        \sum_{i=1}^n x_{1i}    &
        \displaystyle
        \sum_{i=1}^n y_ix_{1i} &
        \displaystyle
        \cdots                 &
        \displaystyle
        \sum_{i=1}^n x_{pi}x_{1i} \\
        \vdots                 &
        \vdots                 &
        \ddots                 &
        \vdots                    \\
        \displaystyle
        \sum_{i=1}^n x_{pi}    &
        \displaystyle
        \sum_{i=1}^n y_ix_{pi} &
        \displaystyle
        \cdots                 &
        \displaystyle
        \sum_{i=1}^n x_{pi}^2
      \end{vmatrix}
    }{
      \begin{vmatrix}
        \displaystyle
        n                         &
        \displaystyle
        \sum_{i=1}^n x_{1i}       &
        \displaystyle
        \cdots                    &
        \displaystyle
        \sum_{i=1}^n x_{pi}         \\
        \displaystyle
        \sum_{i=1}^n x_{1i}       &
        \displaystyle
        \sum_{i=1}^n x_{1i}^2     &
        \displaystyle
        \cdots                    &
        \displaystyle
        \sum_{i=1}^n x_{pi}x_{1i}   \\
        \vdots                    &
        \vdots                    &
        \ddots                    &
        \vdots                      \\
        \displaystyle
        \sum_{i=1}^n x_{pi}       &
        \displaystyle
        \sum_{i=1}^n x_{1i}x_{pi} &
        \displaystyle
        \cdots                    &
        \displaystyle
        \sum_{i=1}^n x_{pi}^2
      \end{vmatrix}
    }
    \\
    \\
    \vdots
    \\
    \\
    \hat{a_p}=
    \frac{
      \begin{vmatrix}
        \displaystyle
        n                         &
        \displaystyle
        \sum_{i=1}^n x_{1i}       &
        \displaystyle
        \cdots                    &
        \displaystyle
        \sum_{i=1}^n y_i            \\
        \displaystyle
        \sum_{i=1}^n x_{1i}       &
        \displaystyle
        \sum_{i=1}^n x_{1i}^2     &
        \displaystyle
        \cdots                    &
        \displaystyle
        \sum_{i=1}^n y_ix_{1i}      \\
        \vdots                    &
        \vdots                    &
        \ddots                    &
        \vdots                      \\
        \displaystyle
        \sum_{i=1}^n x_{pi}       &
        \displaystyle
        \sum_{i=1}^n x_{1i}x_{pi} &
        \displaystyle
        \cdots                    &
        \displaystyle
        \sum_{i=1}^n y_ix_{pi}
      \end{vmatrix}
    }{
      \begin{vmatrix}
        \displaystyle
        n                         &
        \displaystyle
        \sum_{i=1}^n x_{1i}       &
        \displaystyle
        \cdots                    &
        \displaystyle
        \sum_{i=1}^n x_{pi}         \\
        \displaystyle
        \sum_{i=1}^n x_{1i}       &
        \displaystyle
        \sum_{i=1}^n x_{1i}^2     &
        \displaystyle
        \cdots                    &
        \displaystyle
        \sum_{i=1}^n x_{pi}x_{1i}   \\
        \vdots                    &
        \vdots                    &
        \ddots                    &
        \vdots                      \\
        \displaystyle
        \sum_{i=1}^n x_{pi}       &
        \displaystyle
        \sum_{i=1}^n x_{1i}x_{pi} &
        \displaystyle
        \cdots                    &
        \displaystyle
        \sum_{i=1}^n x_{pi}^2
      \end{vmatrix}
    }
  \end{array}
\end{equation}


次に,$x_1, x_2, \cdots, x_p$の\textbf{分散共分散行列(variance-cvariance matrix)}を(11)式のようにおく.
分散共分散行列とは,任意の2つの説明変数間における分散,共分散を行列にまとめたものである.
% (11)
\begin{equation}
  V =
  \begin{pmatrix}
    s_{11} & s_{12} & \cdots & s_{1l} & \cdots & s_{1p} \\
    s_{21} & s_{22} & \cdots & s_{2l} & \cdots & s_{2p} \\
    \vdots & \vdots & \cdots & \vdots & \cdots & \vdots \\
    s_{j1} & s_{j2} & \cdots & s_{jl} & \cdots & s_{jp} \\
    \vdots & \vdots & \cdots & \vdots & \cdots & \vdots \\
    s_{p1} & s_{p2} & \cdots & s_{pl} & \cdots & s_{pp} \\
  \end{pmatrix}
\end{equation}
ここで,$s_{jl}$は(12)式である.
% (12)
\begin{equation}
  s_{jl}=\frac{1}{n}\sum_{i=1}^n (x_{ji}-\bar{x_j})(x_{li}-\bar{x_l}) \; \; (j,l=1, 2, \cdots, p)
\end{equation}
さらに目的変数$y$と説明変数$x_1,x_2,\cdots ,x_p$の共分散を(13)式に示す.
% (13)
\begin{equation}
  \begin{cases}
    \displaystyle
    s_{y1} = \frac{1}{n}\sum_{i=1}^n (y_{i}-\bar{y})(x_{1i}-\bar{x_1}) \\
    \displaystyle
    s_{y2} = \frac{1}{n}\sum_{i=1}^n (y_{i}-\bar{y})(x_{2i}-\bar{x_2}) \\
    \; \; \; \; \; \vdots                                              \\
    \displaystyle
    s_{yp} = \frac{1}{n}\sum_{i=1}^n (y_{i}-\bar{y})(x_{pi}-\bar{x_p})
  \end{cases}
\end{equation}
$\bar{y}$と$\bar{x_j}$は(14)式で表される.
% (14)
\begin{equation}
  \bar{y}=\frac{1}{n}\sum_{i=1}^n y_i , \;\bar{x_j}=\frac{1}{n} \sum_{i=1}^{n} x_{ji} \; \; (j=1, 2, \cdots, p)
\end{equation}

$\hat{a_0},\hat{a_1}, \cdots, \hat{a_p}$を分散・共分散を用いて表すことを考える.
$\hat{a_0},\hat{a_1}, \cdots, \hat{a_p}$は(7)式の正規方程式を満たすから,その一番上の等式より,(15)式を得る.

% (7)
\begin{equation}
  \begin{cases}
    \displaystyle
    na_0+\left(\sum_{i=1}^n x_{1i}\right)a_1+\cdots+\left(\sum_{i=1}^n x_{pi}\right)a_p=\sum_{i=1}^ny_i                                                & \\
    \displaystyle
    \left(\sum_{i=1}^n x_{1i}\right)a_0+\left(\sum_{i=1}^n x_{1i}^2\right)a_1+\cdots+\left(\sum_{i=1}^n x_{pi}x_{1i}\right)a_p=\sum_{i=1}^ny_i x_{1i}  & \\
    \displaystyle
    \; \; \; \vdots                                                                                                                                      \\
    \displaystyle
    \left(\sum_{i=1}^n x_{pi}\right)a_0+\left(\sum_{i=1}^n x_{1i}x_{pi}\right)a_1+\cdots+\left(\sum_{i=1}^n x_{pi}^2\right)a_p=\sum_{i=1}^ny_i x_{pi} &
    \tag{7}
  \end{cases}
\end{equation}


% (15)
\begin{eqnarray}
  \displaystyle
    &&n\hat{a_0}+\left(\sum_{i=1}^n x_{1i}\right)\hat{a_1}+\cdots+\left(\sum_{i=1}^n x_{pi}\right)\hat{a_p}=\sum_{i=1}^ny_i \nonumber \\
    &&\hat{a_0} + \left(\frac{1}{n}\sum_{i=1}^n x_{1i}\right)\hat{a_1}+\cdots+\left(\frac{1}{n}\sum_{i=1}^n x_{pi}\right)\hat{a_p}=\frac{1}{n}\sum_{i=1}^ny_i \nonumber \\
    && \; \; \; \;\hat{a_0}+\hat{a_1}\bar{x_1}+\cdots+\hat{a_p}\bar{x_p} = \bar{y} \nonumber \\
    && \; \; \; \; \hat{a_0} = \bar{y}-(\hat{a_1}\bar{x_1}+\cdots+\hat{a_p}\bar{x_p})
  \end{eqnarray}
また,次の操作がわかりやすくなるように(7)式の左辺を右辺に移行し,変形したものを(16)式に示す.
% (16)
\begin{eqnarray}
  \begin{cases}
    \displaystyle
    \sum_{i=1}^n y_i - \sum_{i=1}^n \hat{a_0} + \left(\sum_{i=1}^n x_{1i} \right)\hat{a_1} + \cdots + \left(\sum_{i=1}^n x_{pi} \right)\hat{a_p} = 0    \\
    \displaystyle
    \left(\sum_{i=1}^n x_{1i} \right)\hat{a_0} + \left(\sum_{i=1}^n x_{1i}^2 \right)\hat{a_1} + \cdots + \left(\sum_{i=1}^n x_{pi}x_{1i} \right)\hat{a_p} = 0    \\
    \; \; \vdots                                                                     \\
    \displaystyle
    \left(\sum_{i=1}^n x_{ji} \right)\hat{a_0} + \left(\sum_{i=1}^n x_{1i}x_{ji} \right)\hat{a_1} + \cdots + \left(\sum_{i=1}^n x_{pi}x_{ji} \right)\hat{a_p} = 0 \\
    \; \; \vdots                                                                     \\
    \displaystyle
    \left(\sum_{i=1}^n x_{pi} \right)\hat{a_0} + \left(\sum_{i=1}^n x_{1i}x_{pi} \right)\hat{a_1} + \cdots + \left(\sum_{i=1}^n x_{pi}^2 \right)\hat{a_p} = 0
  \end{cases}
  \nonumber 
  \\
  \nonumber
  \\
  \begin{cases}
    \displaystyle
    \sum_{i=1}^n \{ y_i-(\hat{a_0}+\hat{a_1}x_{1i}+\cdots+\hat{a_p}x_{pi})\}=0       \\
    \displaystyle
    \sum_{i=1}^n \{ y_i-(\hat{a_0}+\hat{a_1}x_{1i}\cdots+\hat{a_p}x_{pi})\}x_{1i}=0  \\
    \; \; \vdots                                                                     \\
    \displaystyle
    \sum_{i=1}^n \{ y_i-(\hat{a_0}+\hat{a_1}x_{1i}+\cdots+\hat{a_p}x_{pi})\}x_{ji}=0 \\
    \; \; \vdots                                                                     \\
    \displaystyle
    \sum_{i=1}^n \{ y_i-(\hat{a_0}+\hat{a_1}x_{1i}+\cdots+\hat{a_p}x_{pi})\}x_{pi}=0
  \end{cases}
\end{eqnarray}
(15)式を(16)式に代入する.代入し,整理した式を(17)式に示す.
% (17)
\begin{eqnarray}
  \begin{cases}
    \displaystyle
    \sum_{i=1}^n [y_i-\{(\bar{y}-\hat{a_1}\bar{x_1}-\cdots-\hat{a_p}\bar{x_p})+\hat{a_1}x_{1i}+\cdots+\hat{a_p}x_{pi}\}]=0       \\
    \displaystyle
    \sum_{i=1}^n [y_i-\{(\bar{y}-\hat{a_1}\bar{x_1}-\cdots-\hat{a_p}\bar{x_p})+\hat{a_1}x_{1i}\cdots+\hat{a_p}x_{pi}\}]x_{1i}=0  \\
    \; \; \vdots                                                                     \\
    \displaystyle
    \sum_{i=1}^n [y_i-\{(\bar{y}-\hat{a_1}\bar{x_1}-\cdots-\hat{a_p}\bar{x_p})+\hat{a_1}x_{1i}+\cdots+\hat{a_p}x_{pi}\}]x_{ji}=0 \\
    \; \; \vdots                                                                     \\
    \displaystyle
    \sum_{i=1}^n [y_i-\{\bar{y}-\hat{a_1}\bar{x_1}-\cdots-\hat{a_p}\bar{x_p})+\hat{a_1}x_{1i}+\cdots+\hat{a_p}x_{pi}\}]x_{pi}=0
  \end{cases}
  \nonumber
  \\
  \begin{cases}
    \displaystyle
    \sum_{i=1}^n \{ (y_i-\bar{y})-\hat{a_1}(x_{1i}-\bar{x_1})-\cdots-\hat{a_p}(x_{pi}-\bar{x_p}\}=0       \\
    \displaystyle
    \sum_{i=1}^n \{ (y_i-\bar{y})-\hat{a_1}(x_{1i}-\bar{x_1})-\cdots-\hat{a_p}(x_{pi}-\bar{x_p})\}x_{1i}=0 \\
    \; \; \vdots                                                                                           \\
    \displaystyle
    \sum_{i=1}^n \{ (y_i-\bar{y})-\hat{a_1}(x_{1i}-\bar{x_1})-\cdots-\hat{a_p}(x_{pi}-\bar{x_p})\}x_{ji}=0 \\
    \; \; \vdots                                                                                           \\
    \displaystyle
    \sum_{i=1}^n \{ (y_i-\bar{y})-\hat{a_1}(x_{1i}-\bar{x_1})-\cdots-\hat{a_p}(x_{pi}-\bar{x_p})\}x_{pi}=0
  \end{cases}
\end{eqnarray}

説明の簡略化のため,(17)式の連立方程式を上から順番に$0,1, \cdots ,p$でラベル付けする.
$j(j=1,2,\cdots, p)$の方程式から$0$の方程式に$\bar{x_j}$を掛けて,減算すると(18)式のようになる.
% (18)
\begin{eqnarray}
  &&
  \begin{cases}
    \displaystyle
    \sum_{i=1}^n [\{ (y_i-\bar{y})-\hat{a_1}(x_{1i}-\bar{x_1})-\cdots-\hat{a_p}(x_{pi}-\bar{x_p})\}x_{1i}-\{ (y_i-\bar{y})-\hat{a_1}(x_{1i}-\bar{x_1})-\cdots-\hat{a_p}(x_{pi}-\bar{x_p}\}\bar{x_1}]=0 \\
    \; \; \vdots                                                                                           \\
    \displaystyle
    \sum_{i=1}^n [\{ (y_i-\bar{y})-\hat{a_1}(x_{1i}-\bar{x_1})-\cdots-\hat{a_p}(x_{pi}-\bar{x_p})\}x_{ji}-\{ (y_i-\bar{y})-\hat{a_1}(x_{1i}-\bar{x_1})-\cdots-\hat{a_p}(x_{pi}-\bar{x_p}\}\bar{x_j}]=0 \\
    \; \; \vdots                                                                                           \\
    \displaystyle
    \sum_{i=1}^n [\{ (y_i-\bar{y})-\hat{a_1}(x_{1i}-\bar{x_1})-\cdots-\hat{a_p}(x_{pi}-\bar{x_p})\}x_{pi}-\{ (y_i-\bar{y})-\hat{a_1}(x_{1i}-\bar{x_1})-\cdots-\hat{a_p}(x_{pi}-\bar{x_p}\}\bar{x_p}]=0
  \end{cases}
  \nonumber
  \\
  &&
  \begin{cases}
    \displaystyle
    \sum_{i=1}^n \{ (y_i-\bar{y})-\hat{a_1}(x_{1i}-\bar{x_1})-\cdots-\hat{a_p}(x_{pi}-\bar{x_p})\}(x_{1i}-\bar{x_1})=0 \\
    \; \; \vdots                                                                                                       \\
    \displaystyle
    \sum_{i=1}^n \{ (y_i-\bar{y})-\hat{a_1}(x_{1i}-\bar{x_1})-\cdots-\hat{a_p}(x_{pi}-\bar{x_p})\}(x_{ji}-\bar{x_j})=0 \\
    \; \; \vdots                                                                                                       \\
    \displaystyle
    \sum_{i=1}^n \{ (y_i-\bar{y})-\hat{a_1}(x_{1i}-\bar{x_1})-\cdots-\hat{a_p}(x_{pi}-\bar{x_p})\}(x_{pi}-\bar{x_p})=0
  \end{cases}
\end{eqnarray}

(18)式を(11)式と(12)式の記号を用いて整理すると(19)式のようになる.
% (19)
\begin{equation}
  \begin{cases}
    s_{11}\hat{a_1}+s_{12}\hat{a_2}+\cdots+s_{1p}\hat{a_p}=s_{y1} \\
    s_{21}\hat{a_1}+s_{22}\hat{a_2}+\cdots+s_{2p}\hat{a_p}=s_{y2} \\
    \; \; \vdots                                                  \\
    s_{j1}\hat{a_1}+s_{j2}\hat{a_2}+\cdots+s_{jp}\hat{a_p}=s_{yj} \\
    \; \; \vdots                                                  \\
    s_{p1}\hat{a_1}+s_{p2}\hat{a_2}+\cdots+s_{pp}\hat{a_p}=s_{yp}
  \end{cases}
\end{equation}

% (11)
\begin{equation}
  V =
  \begin{pmatrix}
    s_{11} & s_{12} & \cdots & s_{1l} & \cdots & s_{1p} \\
    s_{21} & s_{22} & \cdots & s_{2l} & \cdots & s_{2p} \\
    \vdots & \vdots & \cdots & \vdots & \cdots & \vdots \\
    s_{j1} & s_{j2} & \cdots & s_{jl} & \cdots & s_{jp} \\
    \vdots & \vdots & \cdots & \vdots & \cdots & \vdots \\
    s_{p1} & s_{p2} & \cdots & s_{pl} & \cdots & s_{pp} \\
  \end{pmatrix}
  \tag{11}
\end{equation}

% (12)
\begin{equation}
  s_{jl}=\frac{1}{n}\sum_{i=1}^n (x_{ji}-\bar{x_j})(x_{li}-\bar{x_l}) \; \; (j,l=1, 2, \cdots, p)
  \tag{12}
\end{equation}

(19)式をクラーメルの公式で解くと(20)式のようになる.

\begin{equation}
  \begin{array}{c}
    \hat{a_1} =
    \frac{
      \begin{vmatrix}
        s_{y1}  &
        s_{12 } &
        \cdots  &
        s_{1l}  &
        \cdots  &
        s_{1p}
        \\
        s_{y2}  &
        s_{22 } &
        \cdots  &
        s_{2l}  &
        \cdots  &
        s_{2p}
        \\
        \vdots  &
        \vdots  &
        \ddots  &
        \vdots  &
        \ddots  &
        \vdots  &
        \\
        s_{yj}  &
        s_{j2}  &
        \cdots  &
        s_{jl}  &
        \cdots  &
        s_{jp}    \\
        \vdots  &
        \vdots  &
        \ddots  &
        \vdots  &
        \ddots  &
        \vdots  &
        \\
        s_{yp}  &
        s_{p2 } &
        \cdots  &
        s_{pl}  &
        \cdots  &
        s_{pp}
      \end{vmatrix}
    }{
      \begin{vmatrix}
        s_{11}  &
        s_{12 } &
        \cdots  &
        s_{1l}  &
        \cdots  &
        s_{1p}
        \\
        s_{12}  &
        s_{22 } &
        \cdots  &
        s_{2l}  &
        \cdots  &
        s_{2p}
        \\
        \vdots  &
        \vdots  &
        \ddots  &
        \vdots  &
        \ddots  &
        \vdots  &
        \\
        s_{1j}  &
        s_{j2}  &
        \cdots  &
        s_{jl}  &
        \cdots  &
        s_{jp}    \\
        \vdots  &
        \vdots  &
        \ddots  &
        \vdots  &
        \ddots  &
        \vdots  &
        \\
        s_{1p}  &
        s_{p2 } &
        \cdots  &
        s_{pl}  &
        \cdots  &
        s_{pp}
      \end{vmatrix}
    }

    \\
    \\

    \hat{a_2} =
    \frac{
      \begin{vmatrix}
        s_{11}  &
        s_{y1 } &
        \cdots  &
        s_{1l}  &
        \cdots  &
        s_{1p}
        \\
        s_{12}  &
        s_{y2 } &
        \cdots  &
        s_{2l}  &
        \cdots  &
        s_{2p}
        \\
        \vdots  &
        \vdots  &
        \ddots  &
        \vdots  &
        \ddots  &
        \vdots  &
        \\
        s_{1j}  &
        s_{j2}  &
        \cdots  &
        s_{yj}  &
        \cdots  &
        s_{jp}    \\
        \vdots  &
        \vdots  &
        \ddots  &
        \vdots  &
        \ddots  &
        \vdots  &
        \\
        s_{1p}  &
        s_{yp } &
        \cdots  &
        s_{pl}  &
        \cdots  &
        s_{pp}
      \end{vmatrix}
    }{
      \begin{vmatrix}
        s_{11}  &
        s_{12 } &
        \cdots  &
        s_{1l}  &
        \cdots  &
        s_{1p}
        \\
        s_{12}  &
        s_{22 } &
        \cdots  &
        s_{2l}  &
        \cdots  &
        s_{2p}
        \\
        \vdots  &
        \vdots  &
        \ddots  &
        \vdots  &
        \ddots  &
        \vdots  &
        \\
        s_{1j}  &
        s_{yj}  &
        \cdots  &
        s_{jl}  &
        \cdots  &
        s_{jp}    \\
        \vdots  &
        \vdots  &
        \ddots  &
        \vdots  &
        \ddots  &
        \vdots  &
        \\
        s_{1p}  &
        s_{p2 } &
        \cdots  &
        s_{pl}  &
        \cdots  &
        s_{pp}
      \end{vmatrix}
    }
    \\
    \\
    \vdots
    \\
    \\
    \hat{a_p}=
    \frac{
      \begin{vmatrix}
        s_{11}  &
        s_{12 } &
        \cdots  &
        s_{1l}  &
        \cdots  &
        s_{y1}
        \\
        s_{12}  &
        s_{22 } &
        \cdots  &
        s_{2l}  &
        \cdots  &
        s_{y2}
        \\
        \vdots  &
        \vdots  &
        \ddots  &
        \vdots  &
        \ddots  &
        \vdots  &
        \\
        s_{1j}  &
        s_{j2}  &
        \cdots  &
        s_{jl}  &
        \cdots  &
        s_{yj}    \\
        \vdots  &
        \vdots  &
        \ddots  &
        \vdots  &
        \ddots  &
        \vdots  &
        \\
        s_{1p}  &
        s_{p2 } &
        \cdots  &
        s_{pl}  &
        \cdots  &
        s_{yp}
      \end{vmatrix}
    }{
      \begin{vmatrix}
        s_{11}  &
        s_{12 } &
        \cdots  &
        s_{1l}  &
        \cdots  &
        s_{1p}
        \\
        s_{12}  &
        s_{22 } &
        \cdots  &
        s_{2l}  &
        \cdots  &
        s_{2p}
        \\
        \vdots  &
        \vdots  &
        \ddots  &
        \vdots  &
        \ddots  &
        \vdots  &
        \\
        s_{1j}  &
        s_{j2}  &
        \cdots  &
        s_{jl}  &
        \cdots  &
        s_{jp}    \\
        \vdots  &
        \vdots  &
        \ddots  &
        \vdots  &
        \ddots  &
        \vdots  &
        \\
        s_{1p}  &
        s_{p2 } &
        \cdots  &
        s_{pl}  &
        \cdots  &
        s_{pp}
      \end{vmatrix}
    }
  \end{array}
\end{equation}

$\hat{a_0}$の値は,$\hat{a_j},(j=1,2,\cdots, p)$の値を(14)式に代入することで求めることができる.
\\
\par
よって,(4)式の予測誤差の平方和を最小にする超平面の方程式は(21)式のように表される.
\begin{equation}
  y = \hat{a_0}+\hat{a_1}x_1+\hat{a_2}x_2+\cdots+\hat{a_p}x_p
\end{equation}
(20)式を目的変数$y$の説明変数$x_1,x_2,\cdots, x_p$に対する\textbf{(線形)重回帰式((linear) multiple regression equation)},$\hat{a_j}(j=0,1,\cdots,p)$を\textbf{回帰係数}と呼ぶ.
\par
\  

\section{重回帰式による予測誤差の不偏分散による標準偏差}
予測誤差の不偏分散による標準偏差を求めることで,導出した線形重回帰モデルの予測の良し悪しを知ることができる.
この値が小さければ小さいほど線形重回帰モデルは上手く適合していると言える.
予測誤差の不偏分散の標準偏差を$s_e$とすると,$s_e$は(22)式のように書ける.
\begin{eqnarray}
  s_e&=&\sqrt{\frac{1}{n-(p+1)} \sum_{i=1}^n (\epsilon_i-\bar{\epsilon})^2} \nonumber
  \\
  &=&\sqrt{\frac{1}{n-(p+1)}\sum_{i=1}^n \epsilon_i^2}
\end{eqnarray}

ただし,$\bar{\epsilon}$は(23)式のようになる.
\begin{eqnarray}
  \bar{\epsilon}&=&\frac{1}{n} \sum_{i=1}^n \epsilon_i \nonumber
  \\
  &=&\frac{1}{n} \sum_{i=1}^n  \{ y_i - (\hat{a_0}+ \hat{a_1}x_1 + \cdots + \hat{a_p}x_p)\}  \nonumber
  \\
  &=&\bar{y}-(\hat{a_0}+\hat{a_1}\bar{x_1}+\cdots+\hat{a_p}\bar{x_p}) \nonumber
  \\
  (14)式より,\nonumber
  \\
  &=&0
\end{eqnarray}

ここで,予測誤差の不変分散による標準偏差が(22)式になることを証明する.
ここでは各予測誤差が独立で,$N(0, \sigma^2)$の正規分布に従うことを仮定する.
まず,予測誤差の母集団の母分散$\sigma^2$を標本から予測することを考える.
不偏分散の期待値は母分散になることから,(22)式で与えられた$S_e$を用いて(24)式が定義される.
\begin{equation}
  E[S_e^2]=\frac{1}{n-(p+1)}E\left[\sum_{i=1}^n\epsilon_i\right]=\sigma^2
\end{equation}
(24)式より,(25)式が成り立てば(22)式が成り立つことを言える.
\begin{equation}
  E\left[\sum_{i=1}^n\epsilon_i^2\right]=\{n-(p+1)\}\sigma^2
\end{equation}
\\
説明の簡単化のため,ここからはベクトル表現で話を進める.
ここで,自由に値をとれるデータ数のことを\textbf{自由度(degree of freedom)}と呼ぶ.
(26)式はn個の観測データで構成され,$\boldsymbol{R^n}$の1要素であるとみなせるから,自由度nである.

\begin{equation}
  \boldsymbol{y}=
  \left(
  \begin{array}{c}
      y_1    \\
      y_2    \\
      \vdots \\
      y_n
    \end{array}
  \right)
\end{equation}
(1)式を行列形式で表したものを(27)式に示す.
% (1)
\begin{equation}
  y_i = a_0 + a_1x_{1i}+a_2x_{2i}+\cdots + a_p x_{pi}+\epsilon_i
  \tag{1}
\end{equation}
\begin{eqnarray}
  \boldsymbol{y}&=&\boldsymbol{X}\boldsymbol{a}+\boldsymbol{e}  \nonumber \\
  &=&
  \begin{pmatrix}
    1      & x_{11} & \cdots & x_{p1} \\
    1      & x_{12} & \cdots & x_{p2} \\
    \vdots & \vdots & \vdots & \vdots \\
    1      & x_{1n} & \cdots & x_{pn}
  \end{pmatrix}
  \left(
  \begin{array}{c}
      a_0    \\
      a_1    \\
      \vdots \\
      a_p
    \end{array}
  \right)
  +
  \left(
  \begin{array}{c}
      \epsilon_1 \\
      \epsilon_2 \\
      \vdots     \\
      \epsilon_n
    \end{array}
  \right)
\end{eqnarray}
\begin{equation}
  \boldsymbol{X}:説明変数がp個のとき,n×(p+1)行列 \nonumber
\end{equation}

線形重回帰モデルを求める方法は各予測誤差の平方和を最小にするような$\hat{a_0},\hat{a_1},\cdots,\hat{a_p}$を求めることであった.ここで,(4)式をベクトル表記にしたものを(28)式に示す.
\begin{equation}
  F(a_0,a_1,\cdots,a_p)=\sum_{i=1}^{n}\epsilon_i^2=\sum_{i=1}^{n}\{ y_i - (a_0 + a_1x_{1i}+a_2x_{2i}+\cdots + a_p x_{pi})\}^2
  \tag{4}
\end{equation}

\begin{equation}
  F(\boldsymbol{a})=\sum_{i=1}^{n}\{ y_i - (a_0 + a_1x_{1i}+a_2x_{2i}+\cdots + a_p x_{pi})\}^2=||\boldsymbol{y}-\boldsymbol{Xa}||^2
\end{equation}
(28)式より,最小になるような$F$を求めるということは,$\boldsymbol{y}$とのノルムが最小となる$\boldsymbol{Xa}$(解空間)を求めることと言い換えることができる.$\boldsymbol{Xa}$は$\boldsymbol{a}$の値が不定な自由度$p+1$の超平面である.
超平面$\boldsymbol{Xa}$と$\boldsymbol{y}$のノルムが最小になるのは,$\boldsymbol{y}$から超平面$\boldsymbol{Xa}$へ垂線を下ろしたときであり,この時の予測誤差は(29)式のように書ける.
\begin{equation}
  \boldsymbol{\epsilon}=\boldsymbol{y}-\boldsymbol{X\hat{a}}
\end{equation}
予測誤差$\boldsymbol{\epsilon}$は超平面に下ろした垂線であるから,$\boldsymbol{\epsilon}・\boldsymbol{Xa}=0$である.ここで,超平面$\boldsymbol{Xa}$に直交しており,ベクトル空間の原点を通る超平面$\boldsymbol{\pi}$を考える.すると,$\boldsymbol{\epsilon}$は$\boldsymbol{y}$を超平面$\boldsymbol{\pi}$に射影したものだとみなせる.(図1を参照.)
\par
\begin{figure}[h]
  \begin{center}
    \includegraphics[width=100mm]{重回帰分析.png}
    \caption{ベクトル空間で表した重回帰分析1}
  \end{center}
\end{figure}
次に,予測誤差$\boldsymbol{e}$の$p+1$個の要素が$0$となるような空間の回転を考える.この操作は超平面$\boldsymbol{\pi}$を構成する$p+1$個のベクトルが今考えているベクトル空間$\boldsymbol{R^n}$の$p+1$個の正規直交基底と一致,すなわち内積が$0$となるように空間を回転することである.(超平面$\boldsymbol{\pi}$を構成するベクトルは直交基底である.)	(図2を参照.)
\begin{figure}[h]
  \begin{center}
    \includegraphics[width=100mm]{重回帰分析2.png}
    \caption{ベクトル空間で表した重回帰分析2}
  \end{center}
\end{figure}
変換後の予測誤差$\boldsymbol{\epsilon'}$を(30)式に示す.
\begin{equation}
  \boldsymbol{\epsilon'}=
  \left(
  \begin{array}{c}
      \epsilon_1'         \\
      \epsilon_2'         \\
      \vdots              \\
      \epsilon_{n-(p+1)}' \\
      0                   \\
      \vdots              \\
      0
    \end{array}
  \right)
\end{equation}

また,$\boldsymbol{\epsilon}$の確率密度関数を(31)式に示し,確率密度関数の回転不変性(ベクトルの回転操作によって関数の値が変わらないこと)を証明する.
今,n次元の予測誤差について考えているので,確率密度関数は多変量正規分布になるが,$\epsilon_i$は独立であるから,多変量正規分布を正規分布の積で表すことができる.
\begin{eqnarray}
  \prod_{i=1}^n \frac{1}{\sqrt{2\pi\sigma^2}}exp(-\frac{\epsilon_i^2}{2\sigma^2})
  \displaystyle
  &=&\frac{1}{\sqrt{2\pi\sigma^2}}exp(-\frac{\sum_{i=1}^n\epsilon_i^2}{2\sigma^2}) \nonumber \\
  &=&\frac{1}{\sqrt{2\pi\sigma^2}}exp(-\frac{||\boldsymbol{\epsilon}||^2}{2\sigma^2})
\end{eqnarray}

(31)式より確率密度関数は$\boldsymbol{\epsilon}$のノルムにしか影響を受けない.これより,確率密度関数の回転不変性がいえるとともに,引数にノルムしか取らない期待値にも回転不変性がいえる.よって,(32)式が成り立つ.
\begin{equation}
  E\left[\sum_{i=1}^n\epsilon_i^2\right]=E[||\boldsymbol{\epsilon}||^2]=E[||\boldsymbol{\epsilon}'||^2]
\end{equation}

また,(33)式が成り立つ.
\begin{eqnarray}
  E[||\boldsymbol{\epsilon}'||^2]&=&E[\epsilon_1^2+\epsilon_2^2+\cdots+\epsilon_{n-(p+1)}^2+0+\cdots+0] \nonumber \\
  &=&E[\epsilon_1^2]+E[\epsilon_2^2]+\cdots+E[\epsilon_{n-(p+1)}^2] \nonumber \\
  ここで,E[\epsilon_i^2]&=&V[\epsilon_i]-E[\epsilon_i]^2=V[\epsilon_i]=\sigma^2 \nonumber \\
  &=&\{n-(p+1)\} \sigma^2
\end{eqnarray}

(32)式と(33)式より,(24)式が成り立つ.

\section{重相関係数}
いま,(20)式に基づく予測値を(34)式のようにおく.目的変数の観測値と,予測値および予測誤差を各データに対してまとめておけば表2のように書ける.
\begin{equation}
  Y_i = \hat{a_0}+\hat{a_1}x_{1i}+\hat{a_2}x_{2i}+\cdots+\hat{a_p}x_{pi}, (i=1,2,\cdots,n)
\end{equation}
\begin{table}[htb]
  \begin{center}
    \caption{重回帰分析におけるデータ}
    \begin{tabular}{|c|c|c|}
      \hline
      \rowcolor[rgb]{1.0, 0.8, 0.5}
      観測値$y$ & 予測値$Y$ & 予測誤差$\epsilon$   \\
      \hline
      $y_1$     & $Y_1$     & $\epsilon_1=y_1-Y_1$ \\
      \hline
      $y_2$     & $Y_2$     & $\epsilon_2=y_2-Y_2$ \\
      \hline
      \vdots    & \vdots    & \vdots               \\
      \hline
      $y_n$     & $Y_n$     & $\epsilon_n=y_n-Y_n$ \\
      \hline
    \end{tabular}
  \end{center}
\end{table}

このとき観測値$y$と予測値$Y$の相関係数$r_{yY}$を$y$と$x_1,x_2,\cdots,x_p$の\textbf{重相関係数(multiple correlation coefficient)}と呼び,新しく$r_{y\cdot12\cdots p}$という記号で書くと(35)式のようになる.
\begin{eqnarray}
  r_{y\cdot12\cdots p}&=&\frac{s_{yY}}{\sqrt{s_{yy}s_{YY}}} \nonumber \\
  &=&\frac{\frac{1}{n}\displaystyle \sum_{i=1}^n(y_i-\bar{y})(Y_i-\bar{Y})}{\sqrt{\frac{1}{n}\displaystyle \sum_{i=1}^n (y_i-\bar{y})^2}\sqrt{\frac{1}{n}\displaystyle \sum_{i=1}^n (Y_i-\bar{Y})^2}} \\
  ただし,r_{y\cdot12\cdots p}について, \nonumber \\
  0 \leq &r_{y\cdot12\cdots p}& \leq 1 \nonumber
\end{eqnarray}
また,行列式を用いると$\hat{a_j}$は(36)式のように,$r_{y\cdot12\cdots p}$は(37)式のようにかける.
\begin{equation}
  \hat{a_j}=-\frac{S_{1,j+1}}{S_{11}}
\end{equation}
\begin{equation}
  r_{y\cdot12\cdots p} = \sqrt{1-\frac{S}{S_{yy}S_{11}}}
\end{equation}
ただし,ここで,$S$は(38)式で表される.
\begin{equation}
  S=
  \begin{vmatrix}
    s_{yy} & s_{y1} & s_{y2} & \cdots & \cdots & s_{yp} \\
    s_{1y} & s_{11} & s_{12} & \cdots & \cdots & s_{1p} \\
    s_{2y} & s_{21} & s_{22} & \cdots & \cdots & s_{2p} \\
    \cdots & \cdots & \cdots & \cdots & \cdots & \cdots \\
    \cdots & \cdots & \cdots & \cdots & \cdots & \cdots \\
    s_{py} & s_{p1} & s_{p2} & \cdots & \cdots & s_{pp}
  \end{vmatrix}
\end{equation}
$S_{ij}$は行列式$S$の$i$行$j$列の余因子(i行j列の要素を取り除いて作った行列式に$(-1)^{i+j}$をかけたもの)
\\
(36)式は(20)式の分子にある行列式の$y$との共分散を表す列が一番左に来るように列変形操作を行えば導出できる.
\par
次に(37)式の導出方法を説明する.ここで$s_{yY}$は(39)式のように変形できる.
\begin{eqnarray}
  s_{yY}&=&\frac{1}{n}\sum_{i=1}^n (y_i-\bar{y})(Y_i-\bar{Y}) \nonumber \\
  &=& \frac{1}{n}\sum_{i=1}^n (y_i-\bar{y}) \{(\hat{a_1}x_{1i}+\hat{a_2}x_{2i}+\cdots+\hat{a_p}x_{pi})-(\hat{a_1}\bar{x_1}+\hat{a_2}\bar{x_2}+\cdots+\hat{a_p}\bar{x_p})\} \nonumber \\
  &=&\frac{1}{n}\sum_{i=1}^n(y_i-\bar{y})\{\hat{a_1}(x_{1i}-\bar{x_1})+\hat{a_2}(x_{2i}-\bar{x_2})+\cdots +\hat{a_p}(x_{pi}-\bar{x_p})\}\nonumber \\
  &=&\frac{1}{n}\{\hat{a_1}\sum_{i=1}^n(y_i-\bar{y})(x_{1i}-\bar{x_1})+\hat{a_2}\sum_{i=1}^n(y_i-\bar{y})(x_{2i}-\bar{x_2})+\cdots +\hat{a_p}\sum_{i=1}^n(y_i-\bar{y})(x_{pi}-\bar{x_p}\}\nonumber \\
  &=&\hat{a_1}s_{y1}+\hat{a_2}s_{y2}+\cdots+\hat{a_p}s_{yp} \nonumber \\
  &=&-s_{y1}\frac{S_{12}}{S_{11}}-s_{y2}\frac{S_{13}}{S_{11}}-\cdots-s_{yp}\frac{S_{1,p+1}}{S_{11}} \nonumber \\
  &=&s_{yy}-\frac{S}{S_{11}}
\end{eqnarray}
(39)式の最後の式変形は行列式$S$の余因子展開(40)式を用いた.
\begin{equation}
  S=s_{yy}S_{11}+s_{y1}S_{12}+\cdots+s_{yp}S_{1,p+1}
\end{equation}
一方,$s_{YY}$は(41)式のように変形できる.
\begin{eqnarray}
  s_{YY}&=&\frac{1}{n}\sum_{i=1}^n \{\hat{a_1}(x_{1i}-\bar{x_1})+\hat{a_2}(x_{2i}-\bar{x_2})+\cdots +\hat{a_p}(x_{pi}-\bar{x_p})\}^2
  \nonumber \\
  &=&\hat{a_1}^2s_{11}+\hat{a_2}^2s_{22}+\cdots+\hat{a_p}^2s_{pp}+\cdots+2\hat{a_1}\hat{a_2}s_{12}+2\hat{a_2}{a_3}s_{23}+\cdots \nonumber \\
  &=&\sum_{j=1}^p\sum_{l=1}^p\hat{a_j}\hat{a_l}s_{jl} \nonumber \\
  &=&\sum_{j=1}^p\sum_{l=1}^p\frac{\hat{a_j}\hat{a_l}}{n}\sum_{i=1}^n(x_{ji}-\bar{x_j})(x_{li}-\bar{x_l}) \nonumber \\
  &=&\sum_{j=1}^p\frac{\hat{a_j}}{n}\{\hat{a_1}\sum_{i=1}^n (x_{ji}-\bar{x_j})(x_{1i}-\bar{x_1})+\hat{a_2}\sum_{i=1}^n (x_{ji}-\bar{x_j})(x_{2i}-\bar{x_2}) \nonumber \\
  && \; \; \; \; \; \; \; \; \; \; \; \; \; \; \; \; \; \;+ \cdots+\hat{a_p}\sum_{i=1}^n (x_{ji}-\bar{x_j})(x_{pi}-\bar{x_p})\} \nonumber \\
  &=&\sum_{j=1}^p\frac{\hat{a_j}}{n}\sum_{i=1}^n(x_{ji}-\bar{x_j})\{\hat{a_1}(x_{1i}-\bar{x_1})+\hat{a_2}(x_{2i}-\bar{x_2})+\cdots+\hat{a_p}(x_{pi}-\bar{x_p})\} \nonumber \\
  &=&\sum_{j=1}^p\frac{\hat{a_j}}{n}\sum_{i=1}^n(x_{ji}-\bar{x_j})(Y_{i}-\bar{Y}) \nonumber \\
  &=&\sum_{j=1}^p\frac{\hat{a_j}}{n}\sum_{i=1}^n(x_{ji}-\bar{x_j})(y_{i}-\bar{y}-\epsilon_i) \nonumber \\
  &=&\sum_{j=1}^p\frac{\hat{a_j}}{n}\{\sum_{i=1}^n(x_{ji}-\bar{x_j})(y_{i}-\bar{y})\}-\{\sum_{i=1}^n \epsilon_i(x_{ij}-\bar{xj})\} \nonumber \\
  &=&\sum_{j=1}^p\frac{\hat{a_j}}{n}\sum_{i=1}^n(x_{ji}-\bar{x_j})(y_{i}-\bar{y}) \nonumber \\
  &=&\sum_{j=1}^p\hat{a_j}s_{yj} \nonumber \\
  &=&\frac{1}{n}\sum_{i=1}^n (y_i-\bar{y})\{\hat{a_1}(x_1i-\bar{x_1})+\hat{a_2}(x_2i-\bar{x_2})+\cdots+\hat{a_p}(x_pi-\bar{x_p})\} \nonumber \\
  &=&\frac{1}{n}\sum_{i=1}^n (y_i-\bar{y})(Y_i-\bar{Y}) \nonumber \\
  &=&s_{yY}
\end{eqnarray}
ここで,$\displaystyle -\{\sum_{i=1}^n \epsilon_i(x_{ij}-\bar{xj})\}$の計算結果は説明変数ベクトルと予測誤差ベクトルが
直交であることから,説明変数と予測誤差の積和が$0$となることを用いた.したがって,(42)式が導かれる.
\begin{eqnarray}
  r_{y\cdot12\cdots p} &=& \frac{s_{yY}}{\sqrt{s_{yy}s_{YY}}} \nonumber \\
  &=&\frac{s_{yy}-\frac{S}{S_{11}}}{\sqrt{s_{yy}\left(s_{yy}-\frac{S}{S_{11}}\right)}} \nonumber \\
  &=&\sqrt{1-\frac{S}{s_{yy}s_{11}}}
\end{eqnarray}

% 偏相関係数
\section{偏相関係数}
目的変数$y$に対する説明変数$x_1,x_2\cdots, x_p$のうち,任意の説明変数$x_i$を除く説明変数の回帰を消去することで,
目的変数$y$ととある説明変数$x_i$の相関を考えることができる.
\par
例えば,目的変数$y$および,$x_1$を残りの変数から予測するような2つの重回帰モデルを考える.これを(43)式,(44)式に示す.
\begin{equation}
  y = c_0+c_2x_{2}+\cdots+c_px_{p}+\epsilon
\end{equation}
\begin{equation}
  x_1 = d_0+d_2x_{2}+\cdots+d_px_{p}+\epsilon'
\end{equation}

このとき,最小二乗法によって求めた線形重回帰式は(45)式,(46)式のようになる.
\begin{equation}
  y = \hat{c_0}+\hat{c_2}x_{2}+\cdots+\hat{c_p}x_{p}
\end{equation}
\begin{equation}
  x_{1} = \hat{d_0}+\hat{d_2}x_{2}+\cdots+\hat{d_p}x_{p}
\end{equation}
ただし,$c_i$と$d_i$は(47)式,(48)式のようになる.$S_{22,11}$はSの2行2列の余因子からさらに1行1列の余因子を取ったもの.$S_{22,1(i+1)}$はSの2行2列の余因子からさらに1行j列の余因子を取ったもの.$S_{11},S_{22},S_{11,2(i+1)}$も同様に考える.
上記のように最初にm行m列の余因子を$S$から取り除けば変数mを取り除いた場合の回帰式についての$S$を構成することができる.
\begin{equation}
  \hat{c_i}=-\frac{S_{22,1(i+1)}}{S_{22,11}}
\end{equation}
\begin{equation}
  \hat{d_i}=-\frac{S_{11,2(i+1)}}{S_{11,22}}
\end{equation}
予測誤差を(49)式,(50)式のようにおく.
\begin{equation}
  u_i=y_i-(\hat{c0}+\hat{c_2}x_{2i}+\cdots+\hat{c_p}x_{pi})
\end{equation}
\begin{equation}
  v_i=x_{1i}-(\hat{d0}+\hat{d_2}x_{2i}+\cdots+\hat{d_p}x_{pi})
\end{equation}

変数$u$と$v$の単相関係数は(51)式のようにかける.
\begin{equation}
  r_{y1\cdot23\cdots p}=\frac{s_{uv}}{\sqrt{s_{uu}s_{vv}}}
\end{equation}
ただし,$s_{uu}$,$s_{vv}$,$s_{uv}$,$\bar{u}$,$\bar{v}$は(52)式,(53)式,(54)式,(55)式,(56)式のように定義される.
\begin{equation}
  s_{uu}=\frac{1}{n}\sum_{i=1}^n (u_i-\bar{u})^2
\end{equation}
\begin{equation}
  s_{vv}=\frac{1}{n}\sum_{i=1}^n (v_i-\bar{v})^2
\end{equation}
\begin{equation}
  s_{uv}=\frac{1}{n}\sum_{i=1}^n (u_i-\bar{u})(v_i-\bar{v})
\end{equation}
\begin{equation}
  \bar{u}=\frac{1}{n}\sum_{i=1}^n u_i
\end{equation}
\begin{equation}
  \bar{v}=\frac{1}{n}\sum_{i=1}^n v_i
\end{equation}

(51)式の$r_{y1\cdot23\cdots p}$を$y$および$x_1$から$x_2,x_3,\cdots,x_p$の回帰が消去されたときの\textbf{偏相関係数(partial correlation coefficient)}という.
(49)式,(50)式からわかるように(51)式で示される偏相関係数は$x_2,x_3,\cdots,x_p$の影響を除いた$y$と$x_1$との相関係数を考えられる.同様にして,$y$と$x_2,x_3,\cdots,x_p$の偏相関係数も定義することができる.
\par
また,(39)式に示す行列式$S$と,その余因子を用いると$r_{y1\cdot23\cdots p}$は(57)式のようにも書ける.
\begin{equation}
  S=
  \begin{vmatrix}
    s_{yy} & s_{y1} & s_{y2} & \cdots & \cdots & s_{yp} \\
    s_{1y} & s_{11} & s_{12} & \cdots & \cdots & s_{1p} \\
    s_{2y} & s_{21} & s_{22} & \cdots & \cdots & s_{2p} \\
    \cdots & \cdots & \cdots & \cdots & \cdots & \cdots \\
    \cdots & \cdots & \cdots & \cdots & \cdots & \cdots \\
    s_{py} & s_{p1} & s_{p2} & \cdots & \cdots & s_{pp}
  \end{vmatrix}
  \tag{38}
\end{equation}

\begin{equation}
  r_{y1\cdot23\cdots p}=-\frac{S_{12}}{\sqrt{S_{11}S_{22}}}
\end{equation}
ただし,$S_{11}$,$S_{22}$,$S_{12}$は行列式$S$の1行1列,2行2列,1行2列の余因子.
\par
次に,(57)式の導出方法を説明する.ここで$s_{uu}$は(58)式のように変形できる.
\begin{eqnarray}
  s_{uu}&=&\frac{1}{n}\sum_{i=1}^n(u_i-\bar{u})^2 \nonumber \\
  &=&\frac{1}{n}\sum_{i=1}^n \{(y_i-\bar{y})-(\hat{c_2}(x_{2i}-\bar{x_2})-\hat{c_3}(x_{3i}-\bar{x_3})-\cdots\hat{c_p}(x_{pi}-\bar{x_p}))\}^2  \nonumber \\
  &=&s_{yy}-2\sum_{j=2}^p \hat{c_j}s_{yj} + \sum_{j=2}^p\hat{c_j}\sum_{l=2}^p\hat{c_l}s_{jl} \nonumber \\
  &=&s_{yy}+2\frac{s_{y2}S_{22,13}}{S_{22,11}}+2\frac{s_{y3}S_{22,14}}{S_{22,11}}+\cdots+2\frac{s_{yp}S_{22,1(p+1)}}{S_{22,11}} + \sum_{j=2}^p\hat{c_j}\sum_{l=2}^p\hat{c_l}s_{jl}  \nonumber \\
  &=&s_{yy}+\frac{2}{S_{22,11}}\left\{s_{y2}S_{22,13}+s_{y3}S_{22,14}+\cdots+s_{yp}S_{22,1(p+1)}\right\}  + \sum_{j=2}^p\hat{c_j}\sum_{l=2}^p\hat{c_l}s_{jl} \nonumber \\
  &=&  s_{yy}+\frac{2}{S_{22,11}}\left\{s_{y2}S_{22,13}+s_{y3}S_{22,14}+\cdots+s_{yp}S_{22,1(p+1)}+s_{yy}S_{22,11}-s_{yy}S_{22,11}\right\}  + \sum_{j=2}^p\hat{c_j}\sum_{l=2}^p\hat{c_l}s_{jl}\nonumber \\
  &=& s_{yy}+\frac{2}{S_{22,11}}\left\{S_{22}-s_{yy}S_{22,11}\right\}  + \sum_{j=2}^p\hat{c_j}\sum_{l=2}^p\hat{c_l}s_{jl}\nonumber \\
  &=& s_{yy}+\frac{2}{S_{22,11}}\left\{S_{22}-s_{yy}S_{22,11}\right\}  + \sum_{j=2}^p\hat{c_j}(\hat{c_2}s_{j2}+\hat{c_3}s_{j3}+\cdots+\hat{c_p}s_{jp})\nonumber \\
  &=& s_{yy}+\frac{2}{S_{22,11}}\left\{S_{22}-s_{yy}S_{22,11}\right\}  + \sum_{j=2}^p\hat{c_j}(\hat{c_2}s_{j2}+\hat{c_3}s_{j3}+\cdots+\hat{c_p}s_{jp}+y_{yi}+y_{yi})\nonumber \\
  &=& s_{yy}+\frac{2}{S_{22,11}}\left\{S_{22}-s_{yy}S_{22,11}\right\} \nonumber \\
  && + \sum_{j=2}^p\frac{\hat{c_j}}{n}\{\hat{c_2}\sum_{i=1}^n(x_{ji}-\bar{x_j})(x_{2i}-\bar{x_2})+\hat{c_3}\sum_{i=1}^n (x_{ji}-\bar{x_j})(x_{3i}-\bar{x_3}) \nonumber \\
  && +\cdots+\hat{c_3}\sum_{i=1}^n(x_{ji}-\bar{x_j})(x_{pi}-\bar{x_p})+\sum_{i=1}^n(x_{ji}-\bar{x_j})(xy_{i}-\bar{y})-\sum_{i=1}^n(x_{ji}-\bar{x_j})(y_{i}-\bar{y}))\} \nonumber \\
  &=& s_{yy}+\frac{2}{S_{22,11}}\left\{S_{22}-s_{yy}S_{22,11}\right\} \nonumber \\
  &&+ \sum_{j=2}^p\frac{\hat{c_j}}{n}\sum_{i=1}^n(x_{ji}-\bar{x_j})\{\hat{c_2}(x_{2i}-\bar{x_2})+\hat{c_3}(x_{3i}-\bar{x_3})+\cdots+\hat{c_3}(x_{pi}-\bar{x_p})+(y_{i}-\bar{y})-(y_{i}-\bar{y}))\} \nonumber \\
  &=& s_{yy}+\frac{2}{S_{22,11}}\left\{S_{22}-s_{yy}S_{22,11}\right\} + \sum_{j=2}^p\frac{\hat{c_j}}{n}\sum_{i=1}^n(x_{ji}-\bar{x_j})(y_i-\bar{y})\nonumber \\
  &=& s_{yy}+\frac{2}{S_{22,11}}\left\{S_{22}-s_{yy}S_{22,11}\right\} + \sum_{j=2}^p\hat{c_j}s_{yj}\nonumber \\
  &=& s_{yy}+\frac{2}{S_{22,11}}\left\{S_{22}-s_{yy}S_{22,11}\right\} + \frac{1}{S_{22,11}}\left(s_{y2}S_{22,13}+s_{y3}S_{22,14}+\cdots+s_{yp}S_{22,1(p+1)}\right)\nonumber \\
  &=& s_{yy}+\frac{2}{S_{22,11}}\left\{S_{22}-s_{yy}S_{22,11}\right\} + \frac{1}{S_{22,11}}\left(s_{y2}S_{22,13}+s_{y3}S_{22,14}+\cdots+s_{yp}S_{22,1(p+1)}+s_{yy}S_{22,11}-s_{yy}S_{22,11}\right)\nonumber \\
  &=& s_{yy}+\frac{2}{S_{22,11}}\left\{S_{22}-s_{yy}S_{22,11}\right\} - \frac{1}{S_{22,11}}\left\{S_{22}-s_{yy}S_{22,11}\right\} \nonumber \\
  &=& \frac{S_{22}}{S_{22,11}}
\end{eqnarray}
同様に$s_{vv}$は(59)式のようになる.
\begin{equation}
  S_{vv}=\frac{S_{11}}{S_{11,22}}
\end{equation}
$s_{uv}$は(60)式のように変形できる.
\begin{eqnarray*}
  s_{uv}&=&\frac{1}{n}\sum_{i=1}^n(u_i-\bar{u})(v_i-\bar{v}) \nonumber \\
  &=&\frac{1}{n}\sum_{i=1}^n\left\{y_i-(\hat{c_2}x_{2i}+\hat{c_3}x_{3i}+\cdots+\hat{c_p}x_{pi})-\bar{y}+(\hat{c_2}\bar{x_{2}}+\hat{c_3}\bar{x_{3}}+\cdots+\hat{c_p}\bar{x_{pi}})\right\} \nonumber \\
  &&\left\{x_{1i}-(\hat{d_2}x_{2i}+\hat{d_3}x_{3i}+\cdots+\hat{d_p}x_{pi})-\bar{x_1}+(\hat{d_2}\bar{x_{2}}+\hat{d_3}\bar{x_{3}}+\cdots+\hat{d_p}\bar{x_{pi}})\right\}  \nonumber \\
  &=&\frac{1}{n}\sum_{i=1}^n\left\{ (y_i-\bar{y})-\sum_{j=2}^p \hat{c_j}(x_{ji}-\bar{x_j})\right\} \left\{ (x_{1i}-\bar{x_1})-\sum_{l=2}^p \hat{d_l}(x_{li}-\bar{x_l})\right\} \nonumber \\
  &=&\frac{1}{n}\sum_{i=1}^n\{ (y_i-\bar{y})(x_{1i}-\bar{x_1})-\sum_{j=2}^p\hat{c_j}(x_{1i}-\bar{x_1})(x_{ji}-\bar{x_j}) \nonumber \\
  &&-\sum_{l=2}^p\hat{d_l}(x_{1i}-\bar{x_1})(x_{li}-\bar{x_l})+\sum_{j=2}^p\sum_{l=2}^p\hat{c_j}\hat{d_l}(x_{ji}-\bar{x_j})(x_{li}-\bar{x_l})\} \nonumber \\
  &=&s_{y1}-\sum_{j=2}^p\hat{c_j}s_{1j}-\sum_{l=2}^p\hat{d_l}s_{yl}+\sum_{j=2}^p\sum_{l=2}^p\hat{c_j}\hat{d_l}s_{jl}\nonumber \\
  &=&s_{y1}-\sum_{j=2}^p\frac{\hat{c_j}}{n}\sum_{i=1}^n(x_{1i}-\bar{x_1})(x_{ji}-\bar{x_j})-\sum_{l=2}^p\hat{d_l}s_{yl}+\sum_{j=2}^p\sum_{l=2}^p\hat{c_j}\hat{d_l}s_{jl}\nonumber \\
  &=&s_{y1}-\frac{1}{n} \sum_{i=1}^n\left\{ \hat{c_2}(x_{1i}-\bar{x_1})(x_{2i}-\bar{x_2})+\hat{c_3}(x_{1i}-\bar{x_1})(x_{3i}-\bar{x_3})+\cdots+\hat{c_p}(x_{1i}-\bar{x_1})(x_{pi}-\bar{x_p})\right\} \nonumber \\
  &&-\sum_{l=2}^p\hat{d_l}s_{yl}+\sum_{j=2}^p\sum_{l=2}^p\hat{c_j}\hat{d_l}s_{jl}\nonumber \\
  &=&s_{y1}-\frac{1}{n} \sum_{i=1}^n(x_{1i}-\bar{x_1})\left\{ \hat{c_2}(x_{2i}-\bar{x_2})+\hat{c_3}(x_{3i}-\bar{x_3})+\cdots+\hat{c_p}(x_{pi}-\bar{x_p})\right\} -\sum_{l=2}^p\hat{d_l}s_{yl}+\sum_{j=2}^p\sum_{l=2}^p\hat{c_j}\hat{d_l}s_{jl}\nonumber \\
  &=&s_{y1}-\frac{1}{n} \sum_{i=1}^n(x_{1i}-\bar{x_1})(y_i-\bar{y})-\sum_{l=2}^p\hat{d_l}s_{yl}+\sum_{j=2}^p\sum_{l=2}^p\hat{c_j}\hat{d_l}s_{jl}\nonumber \\
  &=&s_{y1}-s_{y1}-\sum_{l=2}^p\hat{d_l}s_{yl}+\sum_{j=2}^p\sum_{l=2}^p\hat{c_j}\hat{d_l}s_{jl}\nonumber \\
  &=&-\sum_{l=2}^p\hat{d_l}s_{yl}+\sum_{j=2}^p\sum_{l=2}^p\hat{c_j}\hat{d_l}s_{jl}\nonumber \\
\end{eqnarray*}
\begin{eqnarray}
  &=&-\sum_{l=2}^p\frac{\hat{d_j}}{n}\sum_{i=1}^n(y_{i}-\bar{y})(x_{ji}-\bar{x_j})+\sum_{j=2}^p\sum_{l=2}^p\hat{c_j}\hat{d_l}s_{jl}\nonumber \\
  &=&-\frac{1}{n} \sum_{l=1}^n\left\{ \hat{d_2}(y_{i}-\bar{y})(x_{2i}-\bar{x_2})+\hat{d_3}(y_{i}-\bar{y})(x_{3i}-\bar{x_3})+\cdots+\hat{d_p}(y_{i}-\bar{y})(x_{pi}-\bar{x_p})\right\} +\sum_{j=2}^p\sum_{l=2}^p\hat{c_j}\hat{d_l}s_{jl}\nonumber \\
  &=&-\frac{1}{n} \sum_{i=1}^n(y_{i}-\bar{y})\left\{ \hat{d_2}(x_{2i}-\bar{x_2})+\hat{d_3}(x_{3i}-\bar{x_3})+\cdots+\hat{d_p}(x_{pi}-\bar{x_p})\right\} +\sum_{j=2}^p\sum_{l=2}^p\hat{c_j}\hat{d_l}s_{jl}\nonumber \\
  &=&-\frac{1}{n} \sum_{i=1}^n(y_{i}-\bar{y})(x_{1i}-\bar{x_1})+\sum_{j=2}^p\sum_{l=2}^p\hat{c_j}\hat{d_l}s_{jl}\nonumber \\
  &=&-s_{y1}+\sum_{j=2}^p\sum_{l=2}^p\hat{c_j}\hat{d_l}s_{jl}\nonumber \\
  &=&-s_{y1}+\sum_{j=2}^p\frac{\hat{c_j}}{n}\sum_{i=1}^n\left\{ \hat{d_2}(x_{ji}-\bar{x_j})(x_{2j}-\bar{x_2}) + \hat{d_3}(x_{ji}-\bar{x_j})(x_{3j}-\bar{x_3})+\cdots+\hat{d_p}(x_{ji}-\bar{x_j})(x_{pj}-\bar{x_p})\right\}\nonumber \\
  &=&-s_{y1}+\sum_{j=2}^p\frac{\hat{c_j}}{n}\sum_{i=1}^n(x_{ji}-\bar{x_i})\left\{ \hat{d_2}(x_{2j}-\bar{x_2}) + \hat{d_3}(x_{3j}-\bar{x_3})+\cdots+\hat{d_p}(x_{pj}-\bar{x_p})\right\}\nonumber \\
  &=&-s_{y1}+\sum_{j=2}^p\frac{\hat{c_j}}{n}\sum_{i=1}^n(x_{ji}-\bar{x_i})(x_{1i}-\bar{x_1})\nonumber \\
  &=&-s_{y1}+\sum_{j=2}^p\hat{c_j}s_{j1}\nonumber \\
  &=&-s_{y1}-\frac{1}{S_{22,11}}(s_{12}S_{22,13}+s_{13}S_{22,14}+\cdots+s_{1p}S_{22,1p})\nonumber \\
  &=&-s_{y1}-\frac{1}{S_{22,11}}(s_{12}S_{22,13}+s_{13}S_{22,14}+\cdots+s_{1p}S_{22,1p}+s_{1y}S_{22,11}-s_{1y}S_{22,11})\nonumber \\
  &=&-s_{y1}-\frac{1}{S_{22,11}}(S_{12}-s_{1y}S_{22,11})\nonumber \\
  &=&-\frac{S_{12}}{S_{22,11}}
\end{eqnarray}
ここで,$S_{11,22}=S_{22,11}$であるから,(58)式,(59)式,(60)式から(61)式が求まる.
\begin{equation}
  r_{y1\cdot23\cdots p}=\frac{s_{uv}}{\sqrt{s_{uu}s_{vv}}}=\frac{-\frac{S_{12}}{S_{22,11}}}{\sqrt{ \frac{S_{22}}{S_{22,11}} \frac{S_{11}}{S_{11,22}}}}=-\frac{S_{12}}{\sqrt{S_{11}S_{22}}}
\end{equation}

\end{document}